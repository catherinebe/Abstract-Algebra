\documentclass{article}
\usepackage[utf8]{inputenc}
\usepackage{amsmath}
\usepackage{mathrsfs}
\usepackage{mathtools}
\usepackage{amsfonts}
\usepackage{geometry}
\geometry{margin=1in}
% No indentation for entire document
\setlength{\parindent}{4em}
\setlength{\parskip}{1em}
\renewcommand{\baselinestretch}{2.0}

% for url links
\usepackage{url}
% \usepackage{hyperref}
% https://www.latex-tutorial.com/tutorials/hyperlinks/ 

% colored hyperlinks
\usepackage{xcolor}
\usepackage[colorlinks = true,
            linkcolor = blue,
            urlcolor  = purple,
            citecolor = blue
            anchorcolor = blue]{hyperref}

\newcommand{\MYhref}[3][blue]{\href{#2}{\color{#1}{#3}}}%

% Title and ownership
\title{Dummit \& Foot Chapters 1-8 Exercises}
\author{Catherine Berrouet}
\date{October 2020}

\begin{document}

\maketitle

\section*{Authors Note}

In preparation to study mathematics and work within the field of Cryptography, considering research at the doctoral level requires rigor and substantial mathematical stamina as well as proof writing techniques in Algebra and Number Theory.  This documentation will record my personal efforts towards completing an entire collection of proofs for the exercises provided in the Second Edition of the David S. Dummit and Richard M. Foote Abstract Algebra text. 

\noindent \textit{If you are reading this document and have any useful remarks or comments towards corrective edits, please send a direct message to my  \MYhref[purple]{https://www.linkedin.com/in/catherineberrouet2/}{professional LinkedIn account} using the URL link provided in this statement.}  

\noindent Thank you in advance.

\clearpage
\section*{Chapter 1}
\subsection{Determine which of the following binary operations are associative:}
Let G be a group. 

(a) the operation $\star$ on $\mathbb{Z}$ defined by a $\star$ b = a - b.

\noindent Let's first take a look at the general case.  Let z1, z2, z3 be elements in the group G.   Then by the defined operation, $ (z1 \star z2) \star z3$ implies  $ (z1 \star z2) \star z3 = (z1 - z2) \star z3 $.  Since G is a group with closure under addition then we know for some element in G, say g1, we can write g1 = (z1 - z2). Then we have g1 $\star z3 = (g1-z3) = g2$ for some g2 in G.

Now for $ z1 \star (z2 \star z3) $ we have $ z1 \star (z2 - z3) = z1 \star h1$ for some h1 in G, again due to closure under addition operation of G. And so $ z1 \star h1 = z1 - h1 = h2$ for element h2 in G.  Here when G is an arbitrary group, we don't know whether g2 and h2 are equal. However, for our case, we are in the group of Integers so we can use the usual operational computations as follows.  Take 1,2,3 from the set of integers. For the operation $\star$ we have then $ (1\star 2)\star 3 = (-1)\star 3 =-1 - 3 = -4$.\\
Similarly, for $ 1\star(2\star 3) = 1\star (-1) = 1-1 = 0$. Hence, since $ (1\star 2)\star 3 \ne 1\star(2\star 3) $ then it follows that $\star$ is not associate.

(b) the operation $\star$ on $\mathbb{R}$ defined by $a \star b = a+b + ab$.

\noindent Take 1, -1, and 2 for elements in $\mathbb{R}$. 
We have then $ 1\star -1 = 1 + -1 + (1)(-1) = -1$ and so $2\star (1 \star -1) = 2 \star -1 = 2 + 1 + 2(1) = 5 $.  Now for the other arrangement we have $(2\star 1) \star -1 = 5\star -1 = 5 + -1 + 5(-1) = -1$. Thus, clearly $5\ne 1$ and so $\star$ is not associative binary operation on $\mathbb{R}$.

(c) operation $\star$ on $\mathbb{Q}$ defined by $a\star b = \frac{a+b}{5}$ is NOT associative.

\noindent Consider $1,2, 4 \in \mathbb{Q}$ then we have that $$ (1\star 2)\star 4 =  (\frac{1+2}{5}) \star 4 = (\frac{3}{5})\star 4 =   \frac{\frac{3}{5} + 4 }{5} = \frac{\frac{23}{5}}{5} = \frac{23}{25}$$ 

\noindent Rearranging the order of our parenthesis we have  $$ 1\star (2 \star 4) = 1\star (\frac{2+4}{5}) = 1 \star \frac{6}{5} = \frac{1+\frac{6}{5}}{5} =  \frac{\frac{11}{5}}{5} = \frac{55}{5}$$

\noindent Hence, we see that since $ (1\star 2)\star 4 \neq 1\star (2 \star 4) $, we've shown that $\star$ is not associative on $\mathbb{Q}$.

(d) the operation $\star$ on $\mathbb{Z}\times \mathbb{Z}$ defined by $(a,b)\star (c,d) = (ad+bc, bd)$ is associative. 

\noindent Consider $(1,1), (2,0)$ and $(0,1)$ in $\mathbb{Z}\times \mathbb{Z}$.  Then we will show that $$ \bigg ((1,1)\star (2,0)\bigg )\star (0,1) = (1,1)\star \bigg((2,0)\star (0,1)\bigg)$$ since $$ \bigg (1(0)+1(2), 1(0))\bigg )\star (0,1) = (1,1)\star \bigg(2(1)+0(0), 0(1))\bigg) $$
 $$ \bigg (0+2, 0)\bigg )\star (0,1) = (1,1)\star \bigg(2+0, 0)\bigg) $$
 $$ \bigg (2, 0\bigg )\star (0,1) = (1,1)\star \bigg(2, 0\bigg) $$
 $$ \bigg (2(1)+0(0), 0\bigg ) = \bigg(1(0)+1(2), 1(0)\bigg) $$
 $$ \bigg (2, 0\bigg ) = \bigg(2, 0\bigg) $$
 
 \noindent Now to show the operation is associative, we prove for arbitrary elements $(a,b), (c,d), (e,f) \in \mathbb{Z}\times \mathbb{Z}$, we have  $$ \bigg ((a,b)\star (c,d)\bigg )\star (e,f) = (a,b)\star \bigg((c,d)\star (e,f)\bigg)$$ since $$ \bigg (ad+bc, bd \bigg )\star (e,f) = (a,b)\star \bigg(cf+de, df \bigg) $$
 $$ \bigg ((ad+bc)f+bde, bdf)\bigg)  = \bigg(a(df)+b(cf+de), bdf \bigg) $$ 
 then simplifying with closure of distributive laws under usual integer multiplication, we have  $$ \bigg (ad(f)+bc(f)+bde, bdf \bigg ) = \bigg(adf+b(cf)+b(de), bdf \bigg) $$ Hence, 
 $ \bigg (adf+bcf+bde, bdf \bigg ) = \bigg(adf+bcf+bde, bdf\bigg) $.
 Thus, showing associativity.
 
 (e) the operation $\star$ on $\mathbb{Q} - \{0\}$ defined by $a\star b = \frac{a}{b}$. 
 
 \noindent Suppose $q_1, q_2, q_3 \in \mathbb{Q} - \{0\}$ then we have $ q_1\star (q_2\star q_3) = q_1 \star \frac{q_2}{q_3} = \frac{q_1}{\frac{q_2}{q_3}}$ and $ (q_1 \star q_2) \star q_3 = \frac{q_1}{q_2} \star q_3 = \frac{\frac{q_1}{q_2}}{q_3}$. We know that $q_1, q_2, q_3$ cannot be zero, and since $\frac{q_2}{q_3}$ and $\frac{q_1}{q_2}$ are elements in $\mathbb{Q} - \{0\}$ then we have that $ q_1\star (q_2\star q_3)  = (q_1\star q_2) \star q_3 $ 
is true if and only if
 $ \frac{q_1}{\frac{q_2}{q_3}}  = \frac{\frac{q_1}{q_2}}{q_3} \implies \frac{q_1}{1} = \frac{q_1}{q_2} \text{ and } \frac{q_2}{q_3} = \frac{q_3}{1}$   The two equality's imply that  $ 1 = q_2 = q_3$. So it follows that for any arbitrary element $q_1\in \mathbb{Q}-\{0\}$ we would have for eq(1): $ q_1 \star (1 \star 1) = q_1\star 1 = q_1$ and $ (q_1\star 1)\star 1 = q_1 \star 1 = q_1$. So $q_1\star (1\star 1) = (q_1\star 1) \star 1$.  Thereby showing associativity holds for the operation $\star$.
 
  \subsection{Decide which of the binary operations in the preceding exercise are commutative.}
 
 (a) \\
 (b) \\
 (c) \\ 
 (c) \\
 (d) \\
 (e) \\
 
\end{document}
